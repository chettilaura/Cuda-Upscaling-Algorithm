\section{Proposed Method}

In this section we will give to the lecturer a detailed description of the proposed algorithm: 
some implemented procedures will be explained through pseudocode. 
The algorithm works in successive stages as described in the following:

\begin{itemize}
    \item Kernel loading
    \item Image reading
    \item Image cut out
    \item Image enlargement and filling
    \item Image enhancement
\end{itemize}

The algorithm is implemented in C++ and CUDA, and it is divided into two main parts: the first one is the CPU part, which is responsible for the reading of the image
and the kernel loading; the second one is the GPU part, which is responsible for the image cut out, enlargement and filling, and the image enhancement.
The inputs required by the application are based on the commands given by the user through the command line, and they are the following:
\begin{itemize}
    \item The path of the image to be zoomed;
    \item The $-c$ flag, which chooses a custom kernel, followed by the file path of the kernel;
    \item The $-g$ flag, which chooses a gaussian kernel, followed by the length of the kernel and the $\Sigma$ value, mutually exclusive with the previous;
    \item An optional $-v$ flag, which enables the verbose mode;
    \item An optional $-f$ flag, which forces the use of global memory instead of the shared;
    \item The coordinated of the cut out area, in the form of $x,y$;
    \item The width and the height of the cut out area;
    \item The zoom level of the image, which must be an integer bigger than 0.
\end{itemize}

    \subsection{Kernel Loading}
    The algorithm can choose between custom kernels given as input from files, or gaussian ones, which are generated on the fly,
    taking as input the length of the kernel and the $\Sigma$ value. Once generation is done, the kernel is loaded into 
    the constant memory of the GPU.\\ % Add motivations and performance improvements to why this is done

    \subsection{Image Cut-Out}
    The aim of this step is to select the part of the original image that has to be trimmed: 
    the dimension of the cut area is passed through command line and the script subsequently calculates the dimension of image in output.  \\ 
    The measure of the side of the final image is computed choosing the largest multiple, 
    smaller than the smallest dimension between width and height of the original image. These checks are performed in order to avoid the creation of a final 
    image with a dimension that is not a multiple of the side of the cut area.\\
    The function carries out the logic explained down below:

    \begin{equation}
        img_out[tid] = img[starting\_byte + row\_offset + column\_offset]
    \end{equation}
    
    where $tid$ is the thread id, $img\_out$ is the final image, $img$ is the original image, $starting\_byte$ is the starting byte of the cut area, 
    $row\_offset$ is the offset of the row of the pixel to be copied and $column\_offset$ is the offset of the column of the pixel to be copied.
    The implementation is basic, it is a simple copy of the pixels from the original image to the final one, done in parallel using CUDA threads.\\

    \subsection{Image Enlargement and Filling}
    This step is the one that enlarges the image and fills the holes that are created by the zooming process.
    The process begins by creating as many threads as the bytes in the final image, and each one of them computes 
    the value of the pixel to be copied from the original image, so that all holes are filled.\\
    The holes are filled with the help of the pixel replication algorithm. 
    It replicates the neighboring pixels in order to increase them to enlarge the image: it creates new pixels from the already given ones, 
    and it is the most basic technique of implementing the zooming technique, giving as output an image blurred and with lots of artifacts.\\ 
    In the first version of the algorithm, while filling the canvas with the enlarged image, a black border is left around the image,
    which will be used to perform the convolution with the filter, which needs a larger image than the one that is actually returned as output, whichever
    the position from which to start the zooming from.
    The final version of the algorithm, instead, performs the enlargement, and if the cutout is not at the border, the program will use neighboring pixels
    to create a slightly bigger image for the convolution. If the cutout is at the border, instead, the back border will be used just for the border part, 
    leaving a small, unnoticeable shade at the last pixels of the image. This final version allows to have an image which is even more faithful to the original one.\\

    % \begin{equation}
    %     \begin{split}
    %         img\_out &[start\_edge\_coord + r\_dim\_lrgst\_img + c\_dim\_lrgst\_img + col\_offset ] =\\
    %             &= img[r\_offset\_smlr\_img + c\_offset\_smlr\_img + col\_offset\_img]
    %     \end{split}
    % \end{equation}



    \subsection{Image Enhancement}

    This step is based on the convolution operation which is characterized by the type of the odd-length kernel mask used 
    and the memory management configuration set in order to divide the main image into smaller pieces by which divide the work load.   
    
    %[sottosezione sulle masks] 
    The Gaussian kernel mask is one of the two options available in the project, generated on CPU according to the \textit{Gaussian\_Kernel\_Cpu} function: 
    it receives the length and the sigma of the kernel and calculates the mask based on the below given Gaussian Distribution. 
        %mettere qui foto distribuzione gaussiana 
    The \textit{GaussLength} parameter must be an odd value from 3 to 127 sides included and the GaussSigma parameter must be a value from 0.5 onwards side included.
    It is also possible to utilize a custom kernel passed through the command line which has to be specified element by element 
    and it has to be at maximum \textit{MAX\_KERNEL\_DIM}  long: it is set to 127 elements.
    The chosen mask is stored in the \textit{d\_kernel} parameter allocated in the constant memory: having the mask in the constant memory allows us to save time 
    because it is a special type of global memory with a peculiar cache that doesn’t need to do as many coherency tests as the other caches.

    %[sottosezione sul memory management]  
    The convolution operation can be implemented using exclusively the global memory but in some cases, checked in the main file, it can be optimized by exploiting the shared memory through the tiling process.
    The global memory convolution version is implemented in the \textit{globalCudaUpscaling} function and the shared memory one, when available, is performed in the \textit{tilingCudaUpscaling} function: 
    it is also possible to force the global memory version through the boolean variable \textit{forceGlobal} that can be set in the main call. 
    The image is fragmented in order to work on its multiple sections in parallel where each image-fragment is processed in groups of three blocks: one for each color.
    Every block of threads only works on a specific color channel and the convolution is performed between the individual color channel of the single pixel and the mask. 
    Within the GPU kernel call it has to specified the shared memory allocation dimension: there must be a specific shared memory chunk for every block of threads.  
    The shared memory has to be first filled with the color information of the image and than it can be used to perform the convolution between the informations stored and the mask: 
    all the threads of the block participate in filling the memory but afterwards only some of them participate in calculating the convolution with the mask for that block.
    The number of threads set for each block corresponds to the number of bytes allocated for the shared memory: in the project it is set to $bigTileDim*bigTileDim$. 
       %mettere qui schemino disegnato su assegnazione blocchi e colori     




